\documentclass{beamer}

\usepackage{tikz}
\usetikzlibrary{arrows}
\usepackage{hyperref}
\usepackage{fancyvrb}

\newcommand{\bframe}[1]{\begin{frame}[fragile]{#1}}


\newcommand{\point}[2]{
  \draw[fill=black] (#1) circle (0.05) node[above] {\ensuremath{#2}};
}

\newcommand{\arr}[4]{
  \draw[fill=black] (#1) circle (0.05);
  \draw[thick,arrows=->,>=latex] (#1) -- (#2);
  \draw (#1) node[above] {\ensuremath{#3}};
  \draw (#2) node[above] {\ensuremath{#4}};
}

\title{Game Physics Notes 04}
\author{CSCI 321}
\institute{WWU}

\begin{document}\small


\bframe{~}
\titlepage
\end{frame}

\bframe{Intersect two lines}
\begin{itemize}
\item Have to solve for both $s$ and $t$.
\item Should be a better way.
\end{itemize}

\begin{tikzpicture}[>=latex]
%  \draw[lightgray,very thin] (0,0) grid (10,6);
  \draw[dotted] (0,2) -- (8,6);
  \arr {2,3}{4,4}{p}{v};
  \draw[dotted] (10,0) -- (4,6);
  \arr {8,2}{7,3}{q}{w};
  \draw (5,1) node {$p+sv = q+tw$};
\end{tikzpicture}

\end{frame}


\bframe{Intersect two lines}
\begin{itemize}
\item What will be true about $r-q$ if $r$ is on second line?
\end{itemize}

\begin{tikzpicture}[>=latex]
%  \draw[lightgray,very thin] (0,0) grid (10,6);
  \draw[dotted] (0,2) -- (8,6);
  \arr {2,3}{4,4}{p}{v};
  \draw[dotted] (10,0) -- (4,6);
  \arr {8,2}{7,3}{q}{w};
  \arr {8,2}{7,5.5}{}{r-q};
\end{tikzpicture}

\end{frame}

\bframe{Constraint on points in a line}
\begin{itemize}
\item Parametric {\em vs.} Implicit
\end{itemize}

\begin{tikzpicture}[>=latex]
%  \draw[lightgray,very thin] (0,0) grid (10,6);
  \draw[dotted] (0,2) -- (8,6);
  \arr {2,3}{4,4}{p}{v};
  \draw[dotted] (10,0) -- (4,6);
  \arr {8,2}{7,3}{q}{w};
  \arr {8,2}{7,5.5}{}{u};
  \draw (5,1) node {$u\cdot w = 0$};
\end{tikzpicture}

\end{frame}


\bframe{Intersect two lines}

\begin{itemize}
\item use parametric for one and implicit for the other.
\item Solve for $s$ in the resulting equations.
\end{itemize}

\begin{tikzpicture}[>=latex]
%  \draw[lightgray,very thin] (0,0) grid (10,6);
  \draw[dotted] (0,2) -- (8,6);
  \arr {2,3}{4,4}{p}{v};
  \draw[dotted] (10,0) -- (4,6);
  \arr {8,2}{7,3}{q}{w};
  \arr {8,2}{7,5.5}{}{r-q};
  \draw (5,1) node {$(r-q)\cdot w = 0$};
  \draw (5,0) node {$((p+sv)-q)\cdot w = 0$};
\end{tikzpicture}

\end{frame}

\bframe{Intersect line and circle}
\begin{itemize}
\item What is the implicit representation of a circle?
\end{itemize}

\begin{tikzpicture}[>=latex]
%  \draw[lightgray,very thin] (0,0) grid (10,6);
  \draw[dotted] (0,2) -- (8,6);
  \arr {2,3}{4,4}{p}{v};
  \draw (7,4) circle (2);
  \point{7,4}{c};
  \draw (7,4) -- node[above] {$r$} (9,4);
  \point{5.08,4.55}{x_1};
  \point{7.7125,5.87}{x_2};
\end{tikzpicture}

\end{frame}

\bframe{Intersect line and circle}
\begin{itemize}
\item What is the parametric representation of a line?
\end{itemize}

\begin{tikzpicture}[>=latex]
%  \draw[lightgray,very thin] (0,0) grid (10,6);
  \draw[dotted] (0,2) -- (8,6);
  \arr {2,3}{4,4}{p}{v};
  \draw (7,4) circle (2);
  \point{7,4}{c};
  \draw (7,4) -- node[above] {$r$} (9,4);
  \point{5.08,4.55}{x_1};
  \point{7.7125,5.87}{x_2};
  \draw (5,1) node {$|x_1 - c|^2 = r^2$};
  \draw (5,0) node {$|x_2 - c|^2 = r^2$};
\end{tikzpicture}

\end{frame}

\bframe{Intersect line and circle}
\begin{itemize}
\item Solve for $s$ in the quadratic equation.
\item $x_1 = p + s_1v$
\item $x_2 = p + s_2v$
\end{itemize}

\begin{tikzpicture}[>=latex]
%  \draw[lightgray,very thin] (0,0) grid (10,6);
  \draw[dotted] (0,2) -- (8,6);
  \arr {2,3}{4,4}{p}{v};
  \draw (7,4) circle (2);
  \point{7,4}{c};
  \draw (7,4) -- node[above] {$r$} (9,4);
  \point{5.08,4.55}{x_1};
  \point{7.7125,5.87}{x_2};
  \draw (5,1) node {$|(p+sv)-c|^2 = r^2$};
\end{tikzpicture}

\end{frame}

\end{document}
