\documentclass{article}
\usepackage[margin=1in]{geometry}
\title{Modelling, Texturing, Rigging, Skinning, and Animating with
  Blender}
\author{Geoffrey Matthews}
\newcommand{\bi}{\begin{enumerate}}
\newcommand{\ii}{\item}
\newcommand{\ei}{\end{enumerate}}
\newcommand{\arrow}{\ensuremath{\rightarrow}}

\newcommand{\but}[1]{\framebox{\sf #1}}

\begin{document}
\maketitle

\paragraph{Defaults}
\bi
\item Change the \but{Blender Render} dropdown at top to \but{Blender Game}.
\item Enlarge the right hand side panel.
\item In the Properties window, change the \but{Shading} from \but{
  Multitexture} to \but{Glsl}.
\item In the 3D window, change the \but{Viewport shading} from \but{
  Solid} to \but{Texture}.
\item  Enlarge the bottom panel and change the window from \but{
  Timeline} to \but{Logic editor}.
  \item Under the \but{Game} menu, enable \but{Show debug properties}.
\item You may now want to save your defaults, with \but{File \arrow User
  Preferences \arrow Interface \arrow Save User Settings}.

  If you do this,
  you only have to do the above steps once!
\ei

\paragraph{Modelling}
\bi
\item Delete the cube, then add a new one. (Default cube comes with
  lots of stuff attached, which won't be there in general.)
\item Name it \but{Ted}.
\item Select Ted.
\item Press \but{Numpad-.} to center the Ted.
\item Press \but{Numpad-5} to enter orthographic mode.
\item Press \but{Tab} to enter edit mode.
\item Press \but{Z} to enter wireframe mode, or use the \but{Viewport
  shading} popup.
\item Scale Ted in Y with \but{S}, then \but{Y}, then mouse drag.
\item Open the tools shelf with \but{T} toggle.
\item Subdivide  Ted once with the \but{Add \arrow Subdivide} button
  on the tools shelf.
\item Smooth with \but{Deform \arrow Smooth Vertex} on the tools shelf.
\item Go to front view with \but{Numpad-1}.
\item Border select the top side vertices with \but{B}.
\item Extrude them with \but{E}.
\item Shape the head with scaling \but{S}, rotating \but{R}, and
  translating \but{G} (for grab).
\item Extrude and shape a left arm.
\item Add right arm with menu item \but{Mesh \arrow Symmetrize}.
\item Extrude and shape a left leg.
\item Add right leg with menu item \but{Mesh \arrow Symmetrize}.
\item Leave edit mode with \but{Tab}.
\item Smooth the rendering with \but{Shading} set to \but{Smooth} in
  the \but{Tools} shelf.
\item If you like, add a \but{Subdivision surface} modifier in the
  \but{Modifiers} button on the \but{Properties} panel.
\ei

\paragraph{Texturing}
\bi
\item Leave edit mode with \but{Tab} toggle.
\item Add a material to Ted using the \but{Material} button in the
  \but{Properties} panel.
\item Add a texture to Ted using the \but{Texture} button in the
  \but{Properties} panel.
\item Make sure the texture is \but{Image or Movie} and that the
  \but{Mapping} is \but{UV} Coordinates.
\item Create a new texture with the \but{New} button.
\item Name the texture \but{skin}.
\item Drag the corner of the 3d view to creat two side-by-side panels.
\item Change the right one to a \but{UV/Image editor}.
\item Open the \but{skin} image in the image editor.
\item In the 3d view, select Ted, and press \but{Tab} to enter edit
  mode.
\item Press \but{Z} to enter wireframe mode, or use the \but{Viewport
  shading} popup.
\item Go into side view with \but{Numpad-3}.
\item Select the line down the middle with \but{B}, border select.
\item Mark seam with \but{Mesh \arrow Edges \arrow Mark seam}.
\item Select all with \but{A}.
\item Unwrap with \but{Mesh \arrow UV Unwrap \arrow Unwrap}.
\item Back in the UV editor window, change \but{View} mode to
  \but{Paint} mode.
\item Open the tools shelf and paint your image.
\item Save the image with \but{Image \arrow Pack as PNG}.
\item Also click \but{F} next to the image selection popup, to create
  a false user of the texture.
\item Remove the UV window.
\ei

\paragraph{Rigging}
\bi
\item Press \but{Z} to enter wireframe mode, or use the \but{Viewport
  shading} popup.
\item Select Ted.
\item Go to front view with \but{Numpad-1}.
\item Center the 3d cursor with \but{Shift-S \arrow Cursor to selected}
\item Add a bone with \but{Shift-A \arrow Armature \arrow Single
  Bone}.
\item Go into edit mode with \but{Tab} or use the mode popup.
\item Move the bone down in \but{Z} until its root is at hip level.
\item Select the distal end of the bone and extrude a new bone with
  \but{E}.
\item Position end of new bone at the neck.
\item Extrude a head bone from this new bone, position end at top of
  head.
\item In the \but{Tools} shelf, under \but{Options} tab, check
  \but{X-Axis Mirror}.
\item Select the top of the neck bone, and double-extrude with
  \but{Shift-E}.
\item Position the distal end of one of the arm bones at the shoulder
  joint, and let the other one match it at the other shoulder.
\item Extrude this bone again (should extrude a matching bone in the
  other arm), and move its distal end to the end of the arm.
\item Double-extrude from the base of the hips with \but{Shift-E}.
\item Position the distal ends of these bones at the top of the leg.
\item Extrude bones from these to the bottom of the leg.
\item Go back to \but{Texture} in the viewport rendering popup.
\item In the \but{Data} button of the armature select \but{X-Ray}
  under the \but{Display} panel.
\ei 


\paragraph{Skinning}
\bi
\item Go back to \but{Object Mode}
\item Select Ted's mesh.
\item Shift-select the armature.
\item Parent with \but{Control-P}, and \but{Set Parent to \arrow
  Armature deform \arrow With Automatic Weights}
\item Select the armature.
\item Enter pose mode.
\item Pose Ted by rotating bones to see if it worked.
  
\ei

\paragraph{Animating}
\bi
\item Open a \but{Timeline} and a \but{Dope sheet} at the bottom.
\item Press the little red \but{Record} button in the timesheet.
\item Enter pose mode.
\item Select all the bones and make a pose.
\item Scrub the dope sheet forward, make another pose (use \but{Pose
  \arrow Paste X-flipped} when you need it).
\item Scrube forward again, make another pose.
\item When done making poses, unclick record button.
\item Set \but{Start} and \but{End} frames in timeline.
\item Press \but{Play} in timeline and watch your animation.
\item Name your animation.
\item In the game logic, set \but{Actuator} to \but{Play} or
  \but{Loop} your animation.
\item {\em Don't forget to set start and end frames!}
  \ei

\end{document}



  

  
